\documentclass{report}
\include{preamble}
\usepackage{multirow}

\title{\LectureTitle: Computational Homework}

\begin{document}
\maketitle
\newpage
\renewcommand{\thesection}{\arabic{section}}

\section{Exercise 1}
\subsection{a}
My code for this question:
\lstinputlisting[style=Matlab-editor,caption={Continuously Compounded Returns}]{code/ps1a.m}

\begin{figure}[H]
    \centering
    \includegraphics[width=9cm]{figure/Figure1.png}
    \caption{Continuously Compounded Returns: S\&P500}
\end{figure}

\begin{figure}[H]
    \centering
    \includegraphics[width=9cm]{figure/Figure2.png}
    \caption{Continuously Compounded Returns: Dow \_ 30}
\end{figure}

\subsection{b}
My code for this question:
\lstinputlisting[style=Matlab-editor,caption={Summary Stats}]{code/sum_stats.m}

Creating one table containing the output for the two financial asset returns:
\begin{table}[H]
\centering
\begin{tabular}{|c|c|c|c|c|c|c|c|c|c|}
\hline
Name 		& Mean & Median & Max & Min & Std & Skewness & Kurtosis & Pvalue & Jbstat \\ \hline
S\&P 500     & 0.0005 & 0.0005 & 0.0484 & -0.0418 & 0.0082 & -0.5366 & 8.3489 & 0.000 & 936.2640 \\ \hline
Dow\_30  & 0.0006 & 0.0009 & 0.0347 & -0.0507 & 0.0078 &  -0.9166 & 9.5647 & 0.000 & 1466.1400\\ \hline
\end{tabular}
\caption{Summary Stats}
\end{table}

\section{Exercise 1}
\subsection{a}
The estimated parameters $\beta_{0}$ = 0.0005, $\beta_{1}$ =  -0.0883; their standard errors are $6.6236  \times 10^{-5}$; t-statistics are 5909.5000, -61.4643 separately, and R square is 0.0072, the adjusted R square is  0.0058.

\subsection{b}
The t-statistics for $beta_{1}$ is-757.8000 and the critical values for.the test  are -1.9600 and 1.9600 respectively. As you can see, the absolute value of  t-statistics is much larger than 1.9600, so we can reject the null hypothesis that $beta_{1}$ = 1.

\subsection{c}
The estimated parameters  $\beta_{0}$ = 0.0005,  $\beta_{1}$ = -0.0849, $\beta_{2}$ = 0.0820, $\beta_{3}$ = -0.0560; their standard errors  are $8.9401\times 10^{-8}$, 0.0015, 0.0014, 0.0015 separately and their t-statistics are 5587.5000, --58.5220,  -58.7817,  -58.5220; the R square is 0.0147 and adjusted R square is 0.0108.

\subsection{d}
The test value for this regression is 11.2532 and the critical value is 7.8147, since the Test Value is bigger than the critical value, then we canreject the null hypothesis that $\beta_{1}$ = $\beta_{2}$ = $\beta_{3}$ = 0.

\subsection{e}
The absolute of test value for this regression is 10.6534 and the critical value is 5.9915. Then we should reject the hypothesis that $\beta_{1}$ = $\beta_{2}$ = $\beta_{3}$ since the test is greater than the critical value.

My code for this question:
\lstinputlisting[style=Matlab-editor,caption={Estimate Stats}]{code/ps2.m}

\section{Exercise 3}
\subsection{a}
\lstinputlisting[style=Matlab-editor,caption={AR1 Process}]{code/Auto.m}

\subsection{b}
\begin{figure}[H]
    \centering
    \includegraphics[width=9cm]{figure/Figure3.png}
    \caption{AR1 Process}
\end{figure}

\subsection{c}
My code for this question:
\lstinputlisting[style=Matlab-editor,caption={Aucorrelation}]{code/ps3_c.m}

\section{Exercise 4}
\subsection{a}
My code for this question:
\lstinputlisting[style=Matlab-editor,caption={Unusual AR Process}]{code/AR1simU.m}

\subsection{b}
\begin{figure}[H]
    \centering
    \includegraphics[width=9cm]{figure/Figure4.png}
    \caption{AR2 Process}
\end{figure}

\subsection{c}
The unconditional expectation:
$E[Xt] = \frac{ \phi_{0}+ (L+U) / 2}  {1-  \phi_{1} }$

\subsection{d}
My code for this question:
\lstinputlisting[style=Matlab-editor,caption={AR1simR.m}]{code/AR1simR.m}

(1) mean1 = 0.0111

unconmean1 = 0


(2) mean2 = -2.4787

unconmean2 = -2.5000


(3) mean3 = -9.9853

unconmean3 = -10.0000




\end{document}